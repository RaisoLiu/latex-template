\section{Introduction}
我覺得最重要的發現就是不要閉門造車,我在先花費大量的時間、算力與資源之後,發現我的 Reward 都在 10 附近徘徊,大約在 400k 左右的 env steps 之後,就再也沒有進展了,即使我認真的去看 video,發現模型可以做出很帥的切球,但是有時候就是這些切球只要剛好讓對方可以打回來,那這球速就快到很難接到,所以可以穩定取得整場遊戲的勝利,但是沒有辦法穩定不失分,也就到不了最終作業要求的 19 分。

\begin{figure}[htbp]
    \centering
    \includegraphics[width=0.8\textwidth]{figures/IMG_3041 2.jpg}
    \caption{訓練過程中 Episode Steps 的變化。}
    \label{fig:episode_steps}
\end{figure}


在查閱了相關論文之後發現有一種方法 RainbowDQN 可以在 400k 左右的 env steps 之後,達到 19 分,我覺得我當初按照作業要求的實作與 RainbowDQN 的做法主要的差異在於 NoisyLayer 的實作,因為我之前的實作可能由於先學會了切球,然後在快速地獲得 Reward 之後,就卡在這個次優策略中,然而當我實作了 NoisyLayer 之後,發現雖然 Reward 收斂的變慢了,但是 Episode Steps 有顯著的拉高,\label{fig:episode_steps} 中的紅色線。


這代表模型學會了穩定的來回對打,我認為這可能會是一個突破次優策略的關鍵。